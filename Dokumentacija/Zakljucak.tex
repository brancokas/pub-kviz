\chapter{Zaključak i budući rad}
		
		

		
		Aplikaciju smo osmislili kao pomoć kvizašima da pronađu događaje u svojoj blizini koji ih zanimaju i da im olakšamo prijavljivanje vlastitih ekipa na iste.
Također, onima koji nemaju ekipu, a žele se okušati u nadolazećim kvizovima, omogućen je pronalazak nove ekipe kojoj korisnik odgovara svojim definiranim područjima znanja. Korisnik aplikacije može se identificirati kao igrač, sastavljač ili oboje, a sastavljačima aplikacija olakšava objavljivanje i skupljanje pozornosti na kvizove koje kreiraju. \\
Nakon upoznavanja tima i definiranja uloga u timu, počeli smo s radom koji se provodio u dvije faze. U prvoj fazi oblikovali smo potrebnu bazu podataka, dokumentirali funkcionalne i nefunkcionalne zahtjeve koje smo prikazali obrascima uporabe, događaje u aplikaciji smo dokumentirali sekvencijskim dijagramima, a modele objekata koji sudjeluju u aplikaciji dijagramima razreda. Opisali smo arhitekturu sustava i detaljno obrazložili naš cilj i ideju za projekt. Složili smo potreban \textit{frontend} i \textit{backend} koji su zadovoljavali naše zahtjeve za prvu predaju projekta. \\
U drugoj fazi rada svaki je član \textit{frontend} tima radio na svojim dodijeljenim komponentama aplikacije, dok su članovi \textit{backend} tima razvili potporu za spremanje, dohvaćanje i izmjenu podataka. Svi su članovi obogatili svoje iskustvo rada u razvojnom timu i naučili osnove programskog inženjerstva. Vremenski smo bili dobro organizirani, za što je zaslužna česta međusobna komunikacija putem društvenih mreža. Za drugu predaju smo također dokumentirali ispitivanje sustava, \textit{deploy} i dijagrame.\\
Jedno od mogućih budućih funkcionalnih proširenja aplikacije, koje bi bilo dobro implementirati za što bolje korisničko iskustvo, uključuje povećanje maksimalnog broja igrača u ekipi, budući da u stvarnom životu različiti pub kvizovi mogu imati različito definiran maksimalan broj članova. Drugo moguće proširenje je implementacija\textit{ inboxa} među članovima ekipe, ali i među svim korisnicima aplikacije, jer ponekad igrače zanimaju dodatni detalji o događaju i\textit{ inbox} bi olakšao komunikaciju između igrača i sastavljača, kao i ostalo dogovaranje unutar ekipe, što bi korisnicima bilo jako povoljno jer sve vezano za kviz mogu obaviti u jednoj aplikaciji. Također, igračima koji se žele okušati u mnogo kvizova s većim spektrom suigrača, bilo bi dobro omogućiti članstvo u više različitih ekipa.\\


		
		\eject 