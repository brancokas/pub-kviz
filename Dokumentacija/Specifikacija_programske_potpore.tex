\chapter{Specifikacija programske potpore}
		
	\section{Funkcionalni zahtjevi}
			
			\textbf{\textit{dio 1. revizije}}\\
			
			\textit{Navesti \textbf{dionike} koji imaju \textbf{interes u ovom sustavu} ili  \textbf{su nositelji odgovornosti}. To su prije svega korisnici, ali i administratori sustava, naručitelji, razvojni tim.}\\
				
			\textit{Navesti \textbf{aktore} koji izravno \textbf{koriste} ili \textbf{komuniciraju sa sustavom}. Oni mogu imati inicijatorsku ulogu, tj. započinju određene procese u sustavu ili samo sudioničku ulogu, tj. obavljaju određeni posao. Za svakog aktora navesti funkcionalne zahtjeve koji se na njega odnose.}\\
			
			
			\noindent \textbf{Dionici:}
			
			\begin{packed_enum}
				
				\item Naručitelj
				\item Korisnik aplikacije
				\begin{packed_enum}
					\item Igrač kviza
					\item Sastavljač kviza
				\end{packed_enum}				
				\item Administrator
				\item Razvojni tim
				
			\end{packed_enum}
			
			\noindent \textbf{Aktori i njihovi funkcionalni zahtjevi:}
			
			
			\begin{packed_enum}
				\item  \underbar{Neregistrirani/neprijavljeni korisnik (inicijator) može:}
				
				\begin{packed_enum}
					
					\item registrirati se u sustav kao igrač, sastavljač ili oboje istovremeno
					\item pregledati osnovne informacije o događaju (pub kvizu)
					
				\end{packed_enum}
			
				\item  \underbar{Igrač kviza (inicijator) može:}
				
				\begin{packed_enum}
					
					\item prijaviti se u sustav
					\item vidjeti sve objavljene događaje na pregledu "Svi pub kvizovi"
					\item vidjeti događaje na koje je prijavljen na pregledu "Moji pub kvizovi"
					\item pronaći ekipu za kviz
					\item napustiti svoju ekipu
					\item prijaviti svoju ekipu na kviz					
					\item pregledati svoj profil
					\item uređivati svoj profil
					\item pregledati profil sastavljača					 
					\item pretraživati i filtrirati događaje
					\item vidjeti statističke podatke				
							
				\end{packed_enum}
			
				\item  \underbar{Sastavljač kviza (inicijator) može:}
				
				\begin{packed_enum}
					
					\item prijaviti se u sustav
					\item objaviti novi događaj (pub kviz)
					\item vidjeti sve objavljene događaje na pregledu "Svi pub kvizovi"
					\item vidjeti svoje objavljene događaje na pregledu "Moji pub kvizovi"				
					\item pregledati svoj profil
					\item uređivati svoj profil							 
					\item pretraživati i filtrirati događaje
					\item vidjeti statističke podatke	
					
				\end{packed_enum}
			
				\item  \underbar{Administrator (inicijator) može:}
				
				\begin{packed_enum}
					
					\item sve isto što mogu i ostali korisnici sustava
					\item blokirati korisnike koji krše pravila sustava 
					\item odobriti ili zabraniti objavu koju je kreirao sastavljač
					\item brisati objavljene događaje
					\item davati administratorska prava drugim korisnicima
					
				\end{packed_enum}
			
				\item  \underbar{Baza podataka (sudionik) može:}
				
				\begin{packed_enum}
					
					\item pohranjivati sve podatke o korisnicima i njihovim ovlastima
					\item pohranjivati sve podatke o događajima (kvizovima)
					
				\end{packed_enum}
			\end{packed_enum}
			
			\eject 
			
			
				
			\subsection{Obrasci uporabe}
				
				\textbf{\textit{dio 1. revizije}}
				
				\subsubsection{Opis obrazaca uporabe}
					\textit{Funkcionalne zahtjeve razraditi u obliku obrazaca uporabe. Svaki obrazac je potrebno razraditi prema donjem predlošku. Ukoliko u nekom koraku može doći do odstupanja, potrebno je to odstupanje opisati i po mogućnosti ponuditi rješenje kojim bi se tijek obrasca vratio na osnovni tijek.}\\
					

					\noindent \underbar{\textbf{UC$$1$$ -$$ Registracija$$}}
					\begin{packed_item}
	
						\item \textbf{Glavni sudionik: }Korisnik
						\item  \textbf{Cilj: }Stvoriti korisnički račun za pristup sustavu
						\item  \textbf{Sudionici:} Baza podataka
						\item  \textbf{Preduvjet:} -
						\item  \textbf{Opis osnovnog tijeka:}
						
						\item[] \begin{packed_enum}
	
							\item 	Korisnik odabire opciju za registraciju
							\item  Korisnik unosi tražene podatke
							\item Podaci se spremaju u bazu podataka
							\item Korisnika se preusmjerava na početnu stranicu
							
						\end{packed_enum}
						
						\item  \textbf{Opis mogućih odstupanja:}
						
						\item[] \begin{packed_item}
	
							\item[2.a] Korisnik unosi neispravne podatke (zauzeti ili neispravni e-mail ili korisničko ime, unos podataka u nedopuštenom formatu)
							\item[] \begin{packed_enum}
								
								\item Sustav upozorava korisnika na neispravnost unesenih podataka i vraća ga na stranicu za registraciju.
								\item Korisnik mijenja podatke ili odustaje od registracije
								
							\end{packed_enum}
						\end{packed_item}
					\end{packed_item}
				
				\noindent \underbar{\textbf{UC$$2$$ -$$ Prijava u sustav$$}}
				\begin{packed_item}
					
					\item \textbf{Glavni sudionik: }Korisnik
					\item  \textbf{Cilj:} Pristupiti korisničkom sučelju
					\item  \textbf{Sudionici:} Baza podataka
					\item  \textbf{Preduvjet:} Klijent posjeduje korisnički račun
					\item  \textbf{Opis osnovnog tijeka:}
					
					\item[] \begin{packed_enum}
						
						\item Korisnik unosi korisničko ime i lozinku
						\item Podaci prolaze kroz provjeru
						\item Korisnik dobija pristup korisničkom sučelju
					\end{packed_enum}
					
					\item  \textbf{Opis mogućih odstupanja:}
					
					\item[] \begin{packed_item}
						
						\item[2.a] Uneseni su neispravni korisničko ime ili lozinka 
						\item[] \begin{packed_enum}
							
							\item Sustav obavještava korisnika da su uneseni pogrešni podaci i vraća ga na stranicu za prijavu			
						\end{packed_enum}			
					\end{packed_item}
				\end{packed_item}
			
			
				\noindent \underbar{\textbf{UC$$3$$ -$$ Pregled podataka korisničkog profila$$}}
				\begin{packed_item}
					
					\item \textbf{Glavni sudionik: }Korisnik
					\item  \textbf{Cilj:} Pregledati podatke korisničkog profila
					\item  \textbf{Sudionici:} Baza podataka
					\item  \textbf{Preduvjet:} Korisnik je prijavljen u sustav
					\item  \textbf{Opis osnovnog tijeka:}
					
					\item[] \begin{packed_enum}
						
						\item Korisnik odabire opciju „Moj profil“.
						\item Korisnik pregledava podatke profila.
					\end{packed_enum}
					
				\end{packed_item}
			
				
				\noindent \underbar{\textbf{UC$$4$$ -$$ Uređivanje podataka korisničkog profila$$}}
				\begin{packed_item}
					
					\item \textbf{Glavni sudionik: }Korisnik
					\item  \textbf{Cilj:} Urediti podatke korisničkog profila
					\item  \textbf{Sudionici:} Baza podataka
					\item  \textbf{Preduvjet:} Korisnik je prijavljen u sustav
					\item  \textbf{Opis osnovnog tijeka:}
					
					\item[] \begin{packed_enum}
						
						\item Korisnik odabire opciju „Moj profil“
						\item Korisnik odabire opciju za promjenu podataka
						\item Korisnik mijenja odabrane podatke
						\item Korisnik sprema promjene
						\item Baza podataka se ažurira
					\end{packed_enum}
					
					\item  \textbf{Opis mogućih odstupanja:}
					
					\item[] \begin{packed_item}
						
						\item[4.a] Korisnik ne spremi promjene
						\item[] \begin{packed_enum}
							
							\item Promjene se odbacuju
							
						\end{packed_enum}

					\end{packed_item}
				\end{packed_item}
			
				
				\noindent \underbar{\textbf{UC$$5$$ -$$ Kreiranje nadolazećih događaja$$}}
				\begin{packed_item}
					
					\item \textbf{Glavni sudionik: }Sastavljač kviza
					\item  \textbf{Cilj:} Kreiranje novih kvizova
					\item  \textbf{Sudionici:} -
					\item  \textbf{Preduvjet:} Sastavljač kviza je prijavljen u sustav
					\item  \textbf{Opis osnovnog tijeka:}
					
					\item[] \begin{packed_enum}
						
						\item Sastavljač odabire opciju „Kreiraj novi kviz“
						\item Sastavljač popunjava potrebne podatke
						\item Sastavljač sprema promjene
						\item Novi kviz čeka odobrenje administratora
					\end{packed_enum}

				\end{packed_item}
			
			
				\noindent \underbar{\textbf{UC$$6$$ -$$ Odobravanje kreiranih događaja$$}}
				\begin{packed_item}
					
					\item \textbf{Glavni sudionik: }Administrator
					\item  \textbf{Cilj:} Odobriti nove kvizove
					\item  \textbf{Sudionici:} Baza podataka
					\item  \textbf{Preduvjet:} Administrator je prijavljen u sustav
					\item  \textbf{Opis osnovnog tijeka:}
					
					\item[] \begin{packed_enum}
						
						\item Administratoru je dostupan popis novih kvizova koji čekaju odobrenje
						\item Administrator odabire one kvizove koje želi odobriti
						\item Baza podataka se osvježava
						\item Odobreni kvizovi se objavljuju
					\end{packed_enum}
				\end{packed_item}
			
			
				\noindent \underbar{\textbf{UC$$7$$ -$$ Zabrana kreniranih događaja$$}}
				\begin{packed_item}
					
					\item \textbf{Glavni sudionik: }Administrator
					\item  \textbf{Cilj:} Zabraniti neke kvizove
					\item  \textbf{Sudionici:} -
					\item  \textbf{Preduvjet:} Administrator je prijavljen u sustav
					\item  \textbf{Opis osnovnog tijeka:}
					
					\item[] \begin{packed_enum}
						
						\item Administratoru je dostupan popis novih kvizova koji čekaju odobrenje
						\item Administrator odabire one kvizove koje želi zabraniti
						\item Odabrani kvizovi se odbacuju
					\end{packed_enum}
					
				\end{packed_item}
			
			
			
				\noindent \underbar{\textbf{UC$$8$$ -$$ Objava nadolazećih događaja$$}}
				\begin{packed_item}
					
					\item \textbf{Glavni sudionik: }Sastavljač kviza
					\item  \textbf{Cilj:} Objaviti nove kvizove
					\item  \textbf{Sudionici:} Baza podataka, administrator
					\item  \textbf{Preduvjet:} Sastavljač kviza je prijavljen u sustav, administrator je odobrio događaj (kviz) koji je sastavljač kreirao
					\item  \textbf{Opis osnovnog tijeka:}
					
					\item[] \begin{packed_enum}
						
						\item Sastavljač je dobio odobrenje za kreirani kviz
						\item Sastavljač objavljuje kviz koji je vidljiv ostalim korisnicima sustavima
					\end{packed_enum}
					
				
				\end{packed_item}
			
			
				\noindent \underbar{\textbf{UC$$9$$ -$$ Pregled svih objavljenih događaja$$}}
				\begin{packed_item}
					
					\item \textbf{Glavni sudionik: }Korisnik
					\item  \textbf{Cilj:} Pregledati sve događaje koji su objavljeni
					\item  \textbf{Sudionici:} Baza podataka
					\item  \textbf{Preduvjet:} Korisnik je prijavljen u sustav
					\item  \textbf{Opis osnovnog tijeka:}
					
					\item[] \begin{packed_enum}
						
						\item Korisnik odabire opciju „Svi pub kvizovi“
						\item Svi odobreni kvizovi korisniku se prikazuju na zaslonu

					\end{packed_enum}

				\end{packed_item}
			
			
			
			
				\noindent \underbar{\textbf{UC$$10$$ -$$ Pregled svojih objavljenih događaja$$}}
				\begin{packed_item}
					
					\item \textbf{Glavni sudionik: }Sastavljač kviza
					\item  \textbf{Cilj:} Pregledati svoje kreirane kvizove
					\item  \textbf{Sudionici:} Baza podataka
					\item  \textbf{Preduvjet:} Sastavljač kviza je prijavljen u sustav
					\item  \textbf{Opis osnovnog tijeka:}
					
					\item[] \begin{packed_enum}
						
						\item Sastavljač kviza bira opciju „Moji pub kvizovi“
						\item Svi sastavljačevi kreirani kvizovi prikazuju mu se na zaslonu
					\end{packed_enum}
					
				\end{packed_item}
			
			
				
					
				\subsubsection{Dijagrami obrazaca uporabe}
					
					\textit{Prikazati odnos aktora i obrazaca uporabe odgovarajućim UML dijagramom. Nije nužno nacrtati sve na jednom dijagramu. Modelirati po razinama apstrakcije i skupovima srodnih funkcionalnosti.}
				\eject		
				
			\subsection{Sekvencijski dijagrami}
				
				\textbf{\textit{dio 1. revizije}}\\
				
				\textit{Nacrtati sekvencijske dijagrame koji modeliraju najvažnije dijelove sustava (max. 4 dijagrama). Ukoliko postoji nedoumica oko odabira, razjasniti s asistentom. Uz svaki dijagram napisati detaljni opis dijagrama.}
				\eject
	
		\section{Ostali zahtjevi}
		
			\textbf{\textit{dio 1. revizije}}\\
		 
			 \textit{Nefunkcionalni zahtjevi i zahtjevi domene primjene dopunjuju funkcionalne zahtjeve. Oni opisuju \textbf{kako se sustav treba ponašati} i koja \textbf{ograničenja} treba poštivati (performanse, korisničko iskustvo, pouzdanost, standardi kvalitete, sigurnost...). Primjeri takvih zahtjeva u Vašem projektu mogu biti: podržani jezici korisničkog sučelja, vrijeme odziva, najveći mogući podržani broj korisnika, podržane web/mobilne platforme, razina zaštite (protokoli komunikacije, kriptiranje...)... Svaki takav zahtjev potrebno je navesti u jednoj ili dvije rečenice.}
			 
			 
			 
	